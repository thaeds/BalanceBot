\documentclass[12pt,conference,onecolumn,titlepage]{IEEEtran} %This uses the IEEE template

%Set up for inputing Figures
\usepackage[pdftex]{graphicx}
\usepackage{wrapfig}
\usepackage[utf8]{inputenc}
%Include math packages
\usepackage{algorithmic}
\usepackage{pdflscape}
\usepackage{amssymb}
\usepackage{amsmath}
\usepackage{latexsym}
%Added Recommended packages
\usepackage{csquotes}
\usepackage[backend=biber]{biblatex}
\usepackage{hyperref}
\usepackage{cleveref}
%Include packages for line spacing
\usepackage{setspace}
\usepackage{parskip}
\usepackage{indentfirst}
\usepackage{verbatim}
\usepackage{adjustbox}
\usepackage{listings}
\usepackage{subfigure}
\usepackage[T1]{fontenc}
\usepackage[scaled]{beramono}
\newcommand\Small{\fontsize{9}{9.2}\selectfont}
\newcommand*\LSTfont{\Small\ttfamily\SetTracking{encoding=*}{-60}\lsstyle}

\addbibresource{/home/ivan/Dropbox/Documents/Latex_Stuff/BibDB/AllEntries.bib}
%dont think I need to use the graphics path yet
\graphicspath{{/home/ivan/src/Internship_GTRI/Documentation/Images/}
{/home/ivan/src/Internship_GTRI/Documentation/Images/screenshot/}} %Change this to your path!!!

\DeclareGraphicsExtensions{.pdf,.jpeg,.jpg,.png} %Make sure your figure extension is included in this list

%simple command to add a figure 
%\myfigure{address}{caption}{width}{label}
\newcommand{\myfigure}[4]{
  \begin{figure}[h!]
      \centering
      \includegraphics[width=#3\textwidth]{#1}
      \caption{#2}
\label{#4}
    \end{figure}
}

%simple command to add a figure wrapped in text
%\myfigure{image/address}{R_L_PLACEMENT}{width_in_text_width}{caption}{label}
\newcommand{\mywrapfigure}[5]{
  \begin{wrapfigure}{#2}{#3\textwidth}
    \centering
    \includegraphics[width=#3\textwidth]{#1}
    \caption{#4}
\label{#5}
  \end{wrapfigure}
}

\newcommand*{\figuretitle}[1]{%
    {\centering%   <--------  will only affect the title because of the grouping (by the
    \textbf{#1}%              braces before \centering and behind \medskip). If you remove
    \par\medskip}%            these braces the whole body of a {figure} env will be centered.
}

%Set up spacing
\linespread{1.3} %This creates 1.5 spacing
\setlength{\parindent}{12pt} %Sets the length of the indent at the beginning of each paragraph.

%Start the document
\begin{document}

\title{Golem Wing: Resurrection}
\author{Students: Ivan Dario Jimenez\\
Alex Gurney \\
CS3651\\
Date Submitted: \today \\}

\maketitle

\section{Overview}
\label{sec:overview}
%General Stuff

\section{Hardware}
\label{sec:hardware}
%General Introduction to all the physical stuff we had to do to get the GOLEM Wing back working
\subsection{Electrical}
\label{sec:electrical}
%The Iterations of connecting stuff. We should describe the iterative process of finding a better and better way to connect everything and make sure had a hardware stack comaptible with our software
\subsection{Mechanical}
\label{sec:mechanical}
%Just two iterations really. Getting the wheels attached, adding the training wheel and then removing it.
\section{Software}
\label{sec:software}

\subsection{Previous Iterations}
\label{sec:previous-iterations}
                                % Talk about the previous ROS nodes for the accelerometer and any other thing we made but discarded

\subsection{ROS Application Architecture}
\label{sec:ros-appl-arch}

\subsection{Controls}
\label{sec:controls}
% Perhaps talk a b it about our investigation into the controller of the robot, specifically the iteresting soloutions we found for the IK that we implemented for our first presentation and then why we couldn't easily aggregate them into the final controller.
\section{Results}
\label{sec:results}
% A picture of the robot balancing. Maybe even a set of pictures showing how it balanced even though we kicked it. Explain why we got the results we got especially the instabilities and the reason it oscillated. 


\pagebreak
\printbibliography{}
\end{document}